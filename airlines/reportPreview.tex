\documentclass[]{article}
\usepackage{lmodern}
\usepackage{amssymb,amsmath}
\usepackage{ifxetex,ifluatex}
\usepackage{fixltx2e} % provides \textsubscript
\ifnum 0\ifxetex 1\fi\ifluatex 1\fi=0 % if pdftex
  \usepackage[T1]{fontenc}
  \usepackage[utf8]{inputenc}
\else % if luatex or xelatex
  \ifxetex
    \usepackage{mathspec}
  \else
    \usepackage{fontspec}
  \fi
  \defaultfontfeatures{Ligatures=TeX,Scale=MatchLowercase}
\fi
% use upquote if available, for straight quotes in verbatim environments
\IfFileExists{upquote.sty}{\usepackage{upquote}}{}
% use microtype if available
\IfFileExists{microtype.sty}{%
\usepackage{microtype}
\UseMicrotypeSet[protrusion]{basicmath} % disable protrusion for tt fonts
}{}
\usepackage[margin=1in]{geometry}
\usepackage{hyperref}
\hypersetup{unicode=true,
            pdftitle={Sds2 Project},
            pdfauthor={Giacomo Maretto},
            pdfborder={0 0 0},
            breaklinks=true}
\urlstyle{same}  % don't use monospace font for urls
\usepackage{color}
\usepackage{fancyvrb}
\newcommand{\VerbBar}{|}
\newcommand{\VERB}{\Verb[commandchars=\\\{\}]}
\DefineVerbatimEnvironment{Highlighting}{Verbatim}{commandchars=\\\{\}}
% Add ',fontsize=\small' for more characters per line
\usepackage{framed}
\definecolor{shadecolor}{RGB}{248,248,248}
\newenvironment{Shaded}{\begin{snugshade}}{\end{snugshade}}
\newcommand{\KeywordTok}[1]{\textcolor[rgb]{0.13,0.29,0.53}{\textbf{#1}}}
\newcommand{\DataTypeTok}[1]{\textcolor[rgb]{0.13,0.29,0.53}{#1}}
\newcommand{\DecValTok}[1]{\textcolor[rgb]{0.00,0.00,0.81}{#1}}
\newcommand{\BaseNTok}[1]{\textcolor[rgb]{0.00,0.00,0.81}{#1}}
\newcommand{\FloatTok}[1]{\textcolor[rgb]{0.00,0.00,0.81}{#1}}
\newcommand{\ConstantTok}[1]{\textcolor[rgb]{0.00,0.00,0.00}{#1}}
\newcommand{\CharTok}[1]{\textcolor[rgb]{0.31,0.60,0.02}{#1}}
\newcommand{\SpecialCharTok}[1]{\textcolor[rgb]{0.00,0.00,0.00}{#1}}
\newcommand{\StringTok}[1]{\textcolor[rgb]{0.31,0.60,0.02}{#1}}
\newcommand{\VerbatimStringTok}[1]{\textcolor[rgb]{0.31,0.60,0.02}{#1}}
\newcommand{\SpecialStringTok}[1]{\textcolor[rgb]{0.31,0.60,0.02}{#1}}
\newcommand{\ImportTok}[1]{#1}
\newcommand{\CommentTok}[1]{\textcolor[rgb]{0.56,0.35,0.01}{\textit{#1}}}
\newcommand{\DocumentationTok}[1]{\textcolor[rgb]{0.56,0.35,0.01}{\textbf{\textit{#1}}}}
\newcommand{\AnnotationTok}[1]{\textcolor[rgb]{0.56,0.35,0.01}{\textbf{\textit{#1}}}}
\newcommand{\CommentVarTok}[1]{\textcolor[rgb]{0.56,0.35,0.01}{\textbf{\textit{#1}}}}
\newcommand{\OtherTok}[1]{\textcolor[rgb]{0.56,0.35,0.01}{#1}}
\newcommand{\FunctionTok}[1]{\textcolor[rgb]{0.00,0.00,0.00}{#1}}
\newcommand{\VariableTok}[1]{\textcolor[rgb]{0.00,0.00,0.00}{#1}}
\newcommand{\ControlFlowTok}[1]{\textcolor[rgb]{0.13,0.29,0.53}{\textbf{#1}}}
\newcommand{\OperatorTok}[1]{\textcolor[rgb]{0.81,0.36,0.00}{\textbf{#1}}}
\newcommand{\BuiltInTok}[1]{#1}
\newcommand{\ExtensionTok}[1]{#1}
\newcommand{\PreprocessorTok}[1]{\textcolor[rgb]{0.56,0.35,0.01}{\textit{#1}}}
\newcommand{\AttributeTok}[1]{\textcolor[rgb]{0.77,0.63,0.00}{#1}}
\newcommand{\RegionMarkerTok}[1]{#1}
\newcommand{\InformationTok}[1]{\textcolor[rgb]{0.56,0.35,0.01}{\textbf{\textit{#1}}}}
\newcommand{\WarningTok}[1]{\textcolor[rgb]{0.56,0.35,0.01}{\textbf{\textit{#1}}}}
\newcommand{\AlertTok}[1]{\textcolor[rgb]{0.94,0.16,0.16}{#1}}
\newcommand{\ErrorTok}[1]{\textcolor[rgb]{0.64,0.00,0.00}{\textbf{#1}}}
\newcommand{\NormalTok}[1]{#1}
\usepackage{longtable,booktabs}
\usepackage{graphicx,grffile}
\makeatletter
\def\maxwidth{\ifdim\Gin@nat@width>\linewidth\linewidth\else\Gin@nat@width\fi}
\def\maxheight{\ifdim\Gin@nat@height>\textheight\textheight\else\Gin@nat@height\fi}
\makeatother
% Scale images if necessary, so that they will not overflow the page
% margins by default, and it is still possible to overwrite the defaults
% using explicit options in \includegraphics[width, height, ...]{}
\setkeys{Gin}{width=\maxwidth,height=\maxheight,keepaspectratio}
\IfFileExists{parskip.sty}{%
\usepackage{parskip}
}{% else
\setlength{\parindent}{0pt}
\setlength{\parskip}{6pt plus 2pt minus 1pt}
}
\setlength{\emergencystretch}{3em}  % prevent overfull lines
\providecommand{\tightlist}{%
  \setlength{\itemsep}{0pt}\setlength{\parskip}{0pt}}
\setcounter{secnumdepth}{0}
% Redefines (sub)paragraphs to behave more like sections
\ifx\paragraph\undefined\else
\let\oldparagraph\paragraph
\renewcommand{\paragraph}[1]{\oldparagraph{#1}\mbox{}}
\fi
\ifx\subparagraph\undefined\else
\let\oldsubparagraph\subparagraph
\renewcommand{\subparagraph}[1]{\oldsubparagraph{#1}\mbox{}}
\fi

%%% Use protect on footnotes to avoid problems with footnotes in titles
\let\rmarkdownfootnote\footnote%
\def\footnote{\protect\rmarkdownfootnote}

%%% Change title format to be more compact
\usepackage{titling}

% Create subtitle command for use in maketitle
\newcommand{\subtitle}[1]{
  \posttitle{
    \begin{center}\large#1\end{center}
    }
}

\setlength{\droptitle}{-2em}

  \title{Sds2 Project}
    \pretitle{\vspace{\droptitle}\centering\huge}
  \posttitle{\par}
    \author{Giacomo Maretto}
    \preauthor{\centering\large\emph}
  \postauthor{\par}
      \predate{\centering\large\emph}
  \postdate{\par}
    \date{09 gennaio 2019}


\begin{document}
\maketitle

\section{Airlines Fatalities}\label{airlines-fatalities}

\subsection{Introduction}\label{introduction}

In this project we are analyzing data collected from 1976 to 2001 about
airline fatal accidents by the International Civil Aviation Organization
in Montreal , Canada (www.icao.int).

Our goal is to get a good prediction about future fatalitis through a
bayesian approach.

Our data is structured with four columns, year, fatal, miles, rate.

``Passenger miles'' are in units of \(10^{11}\) and the ``accident
rate'' is the number of fatal accidents per \(10^{11}\) passenger miles.

\begin{longtable}[]{@{}cccc@{}}
\toprule
\begin{minipage}[b]{0.09\columnwidth}\centering\strut
year\strut
\end{minipage} & \begin{minipage}[b]{0.10\columnwidth}\centering\strut
fatal\strut
\end{minipage} & \begin{minipage}[b]{0.10\columnwidth}\centering\strut
miles\strut
\end{minipage} & \begin{minipage}[b]{0.10\columnwidth}\centering\strut
rate\strut
\end{minipage}\tabularnewline
\midrule
\endhead
\begin{minipage}[t]{0.09\columnwidth}\centering\strut
1976\strut
\end{minipage} & \begin{minipage}[t]{0.10\columnwidth}\centering\strut
24\strut
\end{minipage} & \begin{minipage}[t]{0.10\columnwidth}\centering\strut
3.863\strut
\end{minipage} & \begin{minipage}[t]{0.10\columnwidth}\centering\strut
6.213\strut
\end{minipage}\tabularnewline
\begin{minipage}[t]{0.09\columnwidth}\centering\strut
1977\strut
\end{minipage} & \begin{minipage}[t]{0.10\columnwidth}\centering\strut
25\strut
\end{minipage} & \begin{minipage}[t]{0.10\columnwidth}\centering\strut
4.3\strut
\end{minipage} & \begin{minipage}[t]{0.10\columnwidth}\centering\strut
5.814\strut
\end{minipage}\tabularnewline
\begin{minipage}[t]{0.09\columnwidth}\centering\strut
1978\strut
\end{minipage} & \begin{minipage}[t]{0.10\columnwidth}\centering\strut
31\strut
\end{minipage} & \begin{minipage}[t]{0.10\columnwidth}\centering\strut
5.027\strut
\end{minipage} & \begin{minipage}[t]{0.10\columnwidth}\centering\strut
6.167\strut
\end{minipage}\tabularnewline
\begin{minipage}[t]{0.09\columnwidth}\centering\strut
1979\strut
\end{minipage} & \begin{minipage}[t]{0.10\columnwidth}\centering\strut
31\strut
\end{minipage} & \begin{minipage}[t]{0.10\columnwidth}\centering\strut
5.481\strut
\end{minipage} & \begin{minipage}[t]{0.10\columnwidth}\centering\strut
5.656\strut
\end{minipage}\tabularnewline
\begin{minipage}[t]{0.09\columnwidth}\centering\strut
1980\strut
\end{minipage} & \begin{minipage}[t]{0.10\columnwidth}\centering\strut
22\strut
\end{minipage} & \begin{minipage}[t]{0.10\columnwidth}\centering\strut
5.814\strut
\end{minipage} & \begin{minipage}[t]{0.10\columnwidth}\centering\strut
3.784\strut
\end{minipage}\tabularnewline
\begin{minipage}[t]{0.09\columnwidth}\centering\strut
1981\strut
\end{minipage} & \begin{minipage}[t]{0.10\columnwidth}\centering\strut
21\strut
\end{minipage} & \begin{minipage}[t]{0.10\columnwidth}\centering\strut
6.033\strut
\end{minipage} & \begin{minipage}[t]{0.10\columnwidth}\centering\strut
3.481\strut
\end{minipage}\tabularnewline
\begin{minipage}[t]{0.09\columnwidth}\centering\strut
1982\strut
\end{minipage} & \begin{minipage}[t]{0.10\columnwidth}\centering\strut
26\strut
\end{minipage} & \begin{minipage}[t]{0.10\columnwidth}\centering\strut
5.877\strut
\end{minipage} & \begin{minipage}[t]{0.10\columnwidth}\centering\strut
4.424\strut
\end{minipage}\tabularnewline
\begin{minipage}[t]{0.09\columnwidth}\centering\strut
1983\strut
\end{minipage} & \begin{minipage}[t]{0.10\columnwidth}\centering\strut
20\strut
\end{minipage} & \begin{minipage}[t]{0.10\columnwidth}\centering\strut
6.223\strut
\end{minipage} & \begin{minipage}[t]{0.10\columnwidth}\centering\strut
3.214\strut
\end{minipage}\tabularnewline
\begin{minipage}[t]{0.09\columnwidth}\centering\strut
1984\strut
\end{minipage} & \begin{minipage}[t]{0.10\columnwidth}\centering\strut
16\strut
\end{minipage} & \begin{minipage}[t]{0.10\columnwidth}\centering\strut
7.433\strut
\end{minipage} & \begin{minipage}[t]{0.10\columnwidth}\centering\strut
2.152\strut
\end{minipage}\tabularnewline
\begin{minipage}[t]{0.09\columnwidth}\centering\strut
1985\strut
\end{minipage} & \begin{minipage}[t]{0.10\columnwidth}\centering\strut
22\strut
\end{minipage} & \begin{minipage}[t]{0.10\columnwidth}\centering\strut
7.107\strut
\end{minipage} & \begin{minipage}[t]{0.10\columnwidth}\centering\strut
3.096\strut
\end{minipage}\tabularnewline
\begin{minipage}[t]{0.09\columnwidth}\centering\strut
1986\strut
\end{minipage} & \begin{minipage}[t]{0.10\columnwidth}\centering\strut
22\strut
\end{minipage} & \begin{minipage}[t]{0.10\columnwidth}\centering\strut
9.1\strut
\end{minipage} & \begin{minipage}[t]{0.10\columnwidth}\centering\strut
2.418\strut
\end{minipage}\tabularnewline
\begin{minipage}[t]{0.09\columnwidth}\centering\strut
1987\strut
\end{minipage} & \begin{minipage}[t]{0.10\columnwidth}\centering\strut
25\strut
\end{minipage} & \begin{minipage}[t]{0.10\columnwidth}\centering\strut
10\strut
\end{minipage} & \begin{minipage}[t]{0.10\columnwidth}\centering\strut
2.5\strut
\end{minipage}\tabularnewline
\begin{minipage}[t]{0.09\columnwidth}\centering\strut
1988\strut
\end{minipage} & \begin{minipage}[t]{0.10\columnwidth}\centering\strut
29\strut
\end{minipage} & \begin{minipage}[t]{0.10\columnwidth}\centering\strut
10.6\strut
\end{minipage} & \begin{minipage}[t]{0.10\columnwidth}\centering\strut
2.736\strut
\end{minipage}\tabularnewline
\begin{minipage}[t]{0.09\columnwidth}\centering\strut
1989\strut
\end{minipage} & \begin{minipage}[t]{0.10\columnwidth}\centering\strut
29\strut
\end{minipage} & \begin{minipage}[t]{0.10\columnwidth}\centering\strut
10.99\strut
\end{minipage} & \begin{minipage}[t]{0.10\columnwidth}\centering\strut
2.639\strut
\end{minipage}\tabularnewline
\begin{minipage}[t]{0.09\columnwidth}\centering\strut
1990\strut
\end{minipage} & \begin{minipage}[t]{0.10\columnwidth}\centering\strut
27\strut
\end{minipage} & \begin{minipage}[t]{0.10\columnwidth}\centering\strut
10.88\strut
\end{minipage} & \begin{minipage}[t]{0.10\columnwidth}\centering\strut
2.482\strut
\end{minipage}\tabularnewline
\begin{minipage}[t]{0.09\columnwidth}\centering\strut
1991\strut
\end{minipage} & \begin{minipage}[t]{0.10\columnwidth}\centering\strut
29\strut
\end{minipage} & \begin{minipage}[t]{0.10\columnwidth}\centering\strut
10.63\strut
\end{minipage} & \begin{minipage}[t]{0.10\columnwidth}\centering\strut
2.727\strut
\end{minipage}\tabularnewline
\begin{minipage}[t]{0.09\columnwidth}\centering\strut
1992\strut
\end{minipage} & \begin{minipage}[t]{0.10\columnwidth}\centering\strut
28\strut
\end{minipage} & \begin{minipage}[t]{0.10\columnwidth}\centering\strut
11.96\strut
\end{minipage} & \begin{minipage}[t]{0.10\columnwidth}\centering\strut
2.342\strut
\end{minipage}\tabularnewline
\begin{minipage}[t]{0.09\columnwidth}\centering\strut
1993\strut
\end{minipage} & \begin{minipage}[t]{0.10\columnwidth}\centering\strut
33\strut
\end{minipage} & \begin{minipage}[t]{0.10\columnwidth}\centering\strut
12.34\strut
\end{minipage} & \begin{minipage}[t]{0.10\columnwidth}\centering\strut
2.674\strut
\end{minipage}\tabularnewline
\begin{minipage}[t]{0.09\columnwidth}\centering\strut
1994\strut
\end{minipage} & \begin{minipage}[t]{0.10\columnwidth}\centering\strut
27\strut
\end{minipage} & \begin{minipage}[t]{0.10\columnwidth}\centering\strut
13.01\strut
\end{minipage} & \begin{minipage}[t]{0.10\columnwidth}\centering\strut
2.075\strut
\end{minipage}\tabularnewline
\begin{minipage}[t]{0.09\columnwidth}\centering\strut
1995\strut
\end{minipage} & \begin{minipage}[t]{0.10\columnwidth}\centering\strut
25\strut
\end{minipage} & \begin{minipage}[t]{0.10\columnwidth}\centering\strut
14.22\strut
\end{minipage} & \begin{minipage}[t]{0.10\columnwidth}\centering\strut
1.758\strut
\end{minipage}\tabularnewline
\begin{minipage}[t]{0.09\columnwidth}\centering\strut
1996\strut
\end{minipage} & \begin{minipage}[t]{0.10\columnwidth}\centering\strut
24\strut
\end{minipage} & \begin{minipage}[t]{0.10\columnwidth}\centering\strut
16.37\strut
\end{minipage} & \begin{minipage}[t]{0.10\columnwidth}\centering\strut
1.466\strut
\end{minipage}\tabularnewline
\begin{minipage}[t]{0.09\columnwidth}\centering\strut
1997\strut
\end{minipage} & \begin{minipage}[t]{0.10\columnwidth}\centering\strut
26\strut
\end{minipage} & \begin{minipage}[t]{0.10\columnwidth}\centering\strut
15.48\strut
\end{minipage} & \begin{minipage}[t]{0.10\columnwidth}\centering\strut
1.679\strut
\end{minipage}\tabularnewline
\begin{minipage}[t]{0.09\columnwidth}\centering\strut
1998\strut
\end{minipage} & \begin{minipage}[t]{0.10\columnwidth}\centering\strut
20\strut
\end{minipage} & \begin{minipage}[t]{0.10\columnwidth}\centering\strut
18.08\strut
\end{minipage} & \begin{minipage}[t]{0.10\columnwidth}\centering\strut
1.106\strut
\end{minipage}\tabularnewline
\begin{minipage}[t]{0.09\columnwidth}\centering\strut
1999\strut
\end{minipage} & \begin{minipage}[t]{0.10\columnwidth}\centering\strut
21\strut
\end{minipage} & \begin{minipage}[t]{0.10\columnwidth}\centering\strut
16.63\strut
\end{minipage} & \begin{minipage}[t]{0.10\columnwidth}\centering\strut
1.263\strut
\end{minipage}\tabularnewline
\begin{minipage}[t]{0.09\columnwidth}\centering\strut
2000\strut
\end{minipage} & \begin{minipage}[t]{0.10\columnwidth}\centering\strut
18\strut
\end{minipage} & \begin{minipage}[t]{0.10\columnwidth}\centering\strut
18.88\strut
\end{minipage} & \begin{minipage}[t]{0.10\columnwidth}\centering\strut
0.954\strut
\end{minipage}\tabularnewline
\begin{minipage}[t]{0.09\columnwidth}\centering\strut
2001\strut
\end{minipage} & \begin{minipage}[t]{0.10\columnwidth}\centering\strut
13\strut
\end{minipage} & \begin{minipage}[t]{0.10\columnwidth}\centering\strut
19.23\strut
\end{minipage} & \begin{minipage}[t]{0.10\columnwidth}\centering\strut
0.676\strut
\end{minipage}\tabularnewline
\bottomrule
\end{longtable}

\begin{center}\rule{0.5\linewidth}{\linethickness}\end{center}

\subsubsection{Following three descriptive time series plot respectively
about the number of fatal accidents, the miles flown and the accident
rate per each
year.}\label{following-three-descriptive-time-series-plot-respectively-about-the-number-of-fatal-accidents-the-miles-flown-and-the-accident-rate-per-each-year.}

\includegraphics{reportPreview_files/figure-latex/unnamed-chunk-3-1.pdf}
\includegraphics{reportPreview_files/figure-latex/unnamed-chunk-3-2.pdf}

\includegraphics{reportPreview_files/figure-latex/unnamed-chunk-4-1.pdf}

\includegraphics{reportPreview_files/figure-latex/unnamed-chunk-5-1.pdf}

started with the simplest model i could immagine, that simply all the
years look the same. This means that the number of fatal accidents in
each year are independend with a \(Poisson(\theta)\) distribution. I set
a non informative gamma prior distribution for \(\theta\), that has
\((\alpha, \beta) = (0.01,0.01)\)

The model for the data is:

where \(\theta\) is the expected number of fatal accidents in an year.
If the prior distribution for \(\theta\) is \((\Gamma(\alpha,\beta))\)
then the posterior distrubution is \(\Gamma(\alpha+n\bar y,\beta+n)\),
where in this case \(n=26\) and \(n\bar y= \sum^{26}_{i=1}y_i=634\)
\[y_i|\theta  \sim Poisson(\theta)\] with:

\begin{verbatim}
        $\theta \sim Gamma(0.01,0.01)$
\end{verbatim}

\subsection{Posterior distribution}\label{posterior-distribution}

The posterior distribution for \(\theta\) is
\(\theta|y \sim \Gamma(634,26)\) and the conditional distribution of
\(\tilde{y}\) (the number of fatal accidents in 2002) is
\(Poisson(\theta)\).

So to simulate values of \(\tilde{y}\) all we need to do is first
generate a realized value form the posterior distribution of \(\theta\)
as the mean. Iterating this process will generate values of
\(\tilde{y}\) from the posterior predictive distribution. What we are
doing here is integrating numerically, using simulation, over the
posterior distrobution in \(\theta\). We can simulate this easily in R.

\begin{Shaded}
\begin{Highlighting}[]
\NormalTok{theta <-}\StringTok{ }\KeywordTok{rgamma}\NormalTok{(}\DecValTok{1000}\NormalTok{, }\DecValTok{634}\NormalTok{, }\DecValTok{26}\NormalTok{ )}
\NormalTok{y.}\DecValTok{2002}\NormalTok{ <-}\StringTok{ }\KeywordTok{rpois}\NormalTok{(}\DecValTok{1000}\NormalTok{,theta)}
\CommentTok{#hist( y.2002 )}
\KeywordTok{plot}\NormalTok{( }\KeywordTok{table}\NormalTok{(y.}\DecValTok{2002}\NormalTok{), }\DataTypeTok{type=}\StringTok{"h"}\NormalTok{, }\DataTypeTok{lwd=}\DecValTok{5}\NormalTok{, }\DataTypeTok{lend=}\DecValTok{2}\NormalTok{, }\DataTypeTok{col=}\KeywordTok{gray}\NormalTok{(}\FloatTok{0.5}\NormalTok{), }\DataTypeTok{bty=}\StringTok{"n"}\NormalTok{, }\DataTypeTok{ylab=}\StringTok{""}\NormalTok{ )}
\end{Highlighting}
\end{Shaded}

\includegraphics{reportPreview_files/figure-latex/unnamed-chunk-6-1.pdf}

\begin{Shaded}
\begin{Highlighting}[]
\KeywordTok{hist}\NormalTok{(airline}\OperatorTok{$}\NormalTok{fatal,}\DataTypeTok{col =} \StringTok{"lightblue"}\NormalTok{)}
\end{Highlighting}
\end{Shaded}

\includegraphics{reportPreview_files/figure-latex/unnamed-chunk-6-2.pdf}

We can specify the model in R using the package rJags.

\subsection{First model}\label{first-model}

\begin{Shaded}
\begin{Highlighting}[]
\KeywordTok{cat}\NormalTok{( }\StringTok{"model \{}
\StringTok{  for( i in 1:I ) \{}
\StringTok{    fatal[i] ~ dpois(mu)}
\StringTok{    \}}
\StringTok{    mu ~ dgamma(0.01,0.01)}
\StringTok{  \}"}\NormalTok{, }\DataTypeTok{file=}\StringTok{"a1.jag"}\NormalTok{ )}

\NormalTok{a1.par <-}\StringTok{ }\KeywordTok{c}\NormalTok{(}\StringTok{"mu"}\NormalTok{,}\StringTok{"fatal[27]"}\NormalTok{)}

\NormalTok{a1.ini <-}\StringTok{ }\KeywordTok{list}\NormalTok{(}\KeywordTok{list}\NormalTok{( }\DataTypeTok{mu=}\DecValTok{22}\NormalTok{ ),}
              \KeywordTok{list}\NormalTok{( }\DataTypeTok{mu=}\DecValTok{23}\NormalTok{ ),}
              \KeywordTok{list}\NormalTok{( }\DataTypeTok{mu=}\DecValTok{24}\NormalTok{ ) )}

\NormalTok{a1.dat <-}\StringTok{ }\KeywordTok{list}\NormalTok{( }\DataTypeTok{fatal =} \KeywordTok{c}\NormalTok{(airline}\OperatorTok{$}\NormalTok{fatal,}\OtherTok{NA}\NormalTok{), }\DataTypeTok{I=}\DecValTok{27}\NormalTok{ )}

\CommentTok{# Model compilation and burn-in}
\NormalTok{a1.mod <-}\StringTok{ }\KeywordTok{jags.model}\NormalTok{( }\DataTypeTok{file =} \StringTok{"a1.jag"}\NormalTok{,}
                    \DataTypeTok{data =}\NormalTok{ a1.dat,}
                    \DataTypeTok{inits =}\NormalTok{ a1.ini,}
                    \DataTypeTok{n.chains =} \DecValTok{3}\NormalTok{,}
                    \DataTypeTok{n.adapt =} \DecValTok{1000}\NormalTok{ )}
\end{Highlighting}
\end{Shaded}

\begin{verbatim}
## Compiling model graph
##    Resolving undeclared variables
##    Allocating nodes
## Graph information:
##    Observed stochastic nodes: 26
##    Unobserved stochastic nodes: 2
##    Total graph size: 30
## 
## Initializing model
\end{verbatim}

\begin{Shaded}
\begin{Highlighting}[]
\CommentTok{# Sampling from the posterior}
\NormalTok{a1.res <-}\StringTok{ }\KeywordTok{coda.samples}\NormalTok{( a1.mod,}
                      \DataTypeTok{var =}\NormalTok{ a1.par,}
                      \DataTypeTok{n.iter =} \DecValTok{10000}\NormalTok{,}
                      \DataTypeTok{thin =} \DecValTok{10}\NormalTok{ )}
\KeywordTok{summary}\NormalTok{( a1.res )}
\end{Highlighting}
\end{Shaded}

\begin{verbatim}
## 
## Iterations = 10:10000
## Thinning interval = 10 
## Number of chains = 3 
## Sample size per chain = 1000 
## 
## 1. Empirical mean and standard deviation for each variable,
##    plus standard error of the mean:
## 
##            Mean     SD Naive SE Time-series SE
## fatal[27] 24.23 5.0284  0.09181        0.09181
## mu        24.36 0.9764  0.01783        0.01720
## 
## 2. Quantiles for each variable:
## 
##            2.5%  25%   50%   75% 97.5%
## fatal[27] 15.00 21.0 24.00 28.00  34.0
## mu        22.44 23.7 24.33 25.03  26.3
\end{verbatim}

\begin{Shaded}
\begin{Highlighting}[]
\NormalTok{theta <-}\StringTok{ }\KeywordTok{rgamma}\NormalTok{(}\DecValTok{6000}\NormalTok{, }\DecValTok{634}\NormalTok{, }\DecValTok{26}\NormalTok{ )}
\NormalTok{y.}\DecValTok{2002}\NormalTok{ <-}\StringTok{ }\KeywordTok{rpois}\NormalTok{(}\DecValTok{6000}\NormalTok{,theta)}
\KeywordTok{plot}\NormalTok{( }\DataTypeTok{main =} \StringTok{'Posterior predictive distribution of y in 2002'}\NormalTok{,}\KeywordTok{table}\NormalTok{(y.}\DecValTok{2002}\NormalTok{), }\DataTypeTok{type=}\StringTok{"h"}\NormalTok{, }\DataTypeTok{lwd=}\DecValTok{5}\NormalTok{, }\DataTypeTok{lend=}\DecValTok{2}\NormalTok{, }\DataTypeTok{col=}\KeywordTok{gray}\NormalTok{(}\FloatTok{0.2}\NormalTok{), }\DataTypeTok{bty=}\StringTok{"n"}\NormalTok{, }\DataTypeTok{ylab=}\StringTok{""}\NormalTok{, }\DataTypeTok{xlim=}\KeywordTok{c}\NormalTok{(}\DecValTok{5}\NormalTok{,}\DecValTok{60}\NormalTok{) )}
\NormalTok{tpr <-}\StringTok{ }\KeywordTok{table}\NormalTok{( }\KeywordTok{as.matrix}\NormalTok{( a1.res[,}\StringTok{"fatal[27]"}\NormalTok{] ) )}
\KeywordTok{points}\NormalTok{( }\KeywordTok{as.numeric}\NormalTok{(}\KeywordTok{names}\NormalTok{(tpr))}\OperatorTok{+}\FloatTok{0.4}\NormalTok{, tpr, }\DataTypeTok{type=}\StringTok{"h"}\NormalTok{, }\DataTypeTok{col=}\StringTok{"red"}\NormalTok{, }\DataTypeTok{lwd=}\DecValTok{4}\NormalTok{ )}
\KeywordTok{legend}\NormalTok{(}\StringTok{"topright"}\NormalTok{, }\KeywordTok{c}\NormalTok{(}\StringTok{'Directly simulated'}\NormalTok{,}\StringTok{'posterior values from BUGS output'}\NormalTok{),}\DataTypeTok{col=}\KeywordTok{c}\NormalTok{(}\KeywordTok{gray}\NormalTok{(}\FloatTok{0.2}\NormalTok{),}\StringTok{'red'}\NormalTok{),}\DataTypeTok{lwd =} \DecValTok{5}\NormalTok{)}
\end{Highlighting}
\end{Shaded}

\includegraphics{reportPreview_files/figure-latex/unnamed-chunk-7-1.pdf}
\#\# Diagnostics for model 1

\begin{Shaded}
\begin{Highlighting}[]
\NormalTok{result =}\StringTok{ }\KeywordTok{ggs}\NormalTok{(a1.res)}
\KeywordTok{ggs_density}\NormalTok{(result)}
\end{Highlighting}
\end{Shaded}

\includegraphics{reportPreview_files/figure-latex/unnamed-chunk-8-1.pdf}

\begin{Shaded}
\begin{Highlighting}[]
\KeywordTok{ggs_traceplot}\NormalTok{(result)}
\end{Highlighting}
\end{Shaded}

\includegraphics{reportPreview_files/figure-latex/unnamed-chunk-8-2.pdf}

\begin{Shaded}
\begin{Highlighting}[]
\KeywordTok{ggs_running}\NormalTok{(result)}
\end{Highlighting}
\end{Shaded}

\includegraphics{reportPreview_files/figure-latex/unnamed-chunk-8-3.pdf}

\begin{Shaded}
\begin{Highlighting}[]
\KeywordTok{ggs_autocorrelation}\NormalTok{(result)}
\end{Highlighting}
\end{Shaded}

\includegraphics{reportPreview_files/figure-latex/unnamed-chunk-8-4.pdf}

\subsection{Model 2}\label{model-2}

A more natural model is the multiplicative one

\[log(E(y(t)|t,m(t))) = \alpha + \beta t + log(m(t))\]

with :

\[y_i \sim Pois(\mu)\] and \[\mu = exp(\alpha + \beta * year ) * miles\]

with uninformative ppriors:

\(\alpha \sim Norm(0,0.000001)\) \(\beta \sim Norm(0,0.000001)\)

\begin{Shaded}
\begin{Highlighting}[]
\KeywordTok{summary}\NormalTok{(glm2 <-}\StringTok{ }\KeywordTok{glm}\NormalTok{( fatal }\OperatorTok{~}\StringTok{ }\KeywordTok{I}\NormalTok{(year}\OperatorTok{-}\DecValTok{1985}\NormalTok{) }\OperatorTok{+}\StringTok{ }\KeywordTok{offset}\NormalTok{(}\KeywordTok{log}\NormalTok{(miles)),}\DataTypeTok{family=}\NormalTok{poisson,}\DataTypeTok{data=}\NormalTok{airline)) }
\end{Highlighting}
\end{Shaded}

\begin{verbatim}
## 
## Call:
## glm(formula = fatal ~ I(year - 1985) + offset(log(miles)), family = poisson, 
##     data = airline)
## 
## Deviance Residuals: 
##     Min       1Q   Median       3Q      Max  
## -2.0782  -0.7953   0.1626   0.7190   1.9369  
## 
## Coefficients:
##                 Estimate Std. Error z value Pr(>|z|)    
## (Intercept)     1.176111   0.043200   27.23   <2e-16 ***
## I(year - 1985) -0.068742   0.005394  -12.74   <2e-16 ***
## ---
## Signif. codes:  0 '***' 0.001 '**' 0.01 '*' 0.05 '.' 0.1 ' ' 1
## 
## (Dispersion parameter for poisson family taken to be 1)
## 
##     Null deviance: 182.628  on 25  degrees of freedom
## Residual deviance:  22.545  on 24  degrees of freedom
## AIC: 157.02
## 
## Number of Fisher Scoring iterations: 4
\end{verbatim}

which shows that rates decrease about 7\% per year.

We can now fit a model using JAGS

\begin{Shaded}
\begin{Highlighting}[]
\KeywordTok{cat}\NormalTok{( }\StringTok{"model}
\StringTok{ \{}
\StringTok{ for( i in 1:I )}
\StringTok{ \{}
\StringTok{ mu[i] <- exp( alpha + beta*(i-10) ) * miles[i]}
\StringTok{ fatal[i] ~ dpois( mu[i] )}
\StringTok{ \}}
\StringTok{ alpha ~ dnorm(0,0.000001)}
\StringTok{ beta ~ dnorm(0,0.000001)}
\StringTok{ \}"}\NormalTok{,}
 \DataTypeTok{file=}\StringTok{"a2.jag"}\NormalTok{ )}
\NormalTok{ a2.ini <-}\StringTok{ }\KeywordTok{list}\NormalTok{( }\KeywordTok{list}\NormalTok{( }\DataTypeTok{alpha=}\FloatTok{1.0}\NormalTok{, }\DataTypeTok{beta=}\OperatorTok{-}\FloatTok{0.05}\NormalTok{ ),}
 \KeywordTok{list}\NormalTok{( }\DataTypeTok{alpha=}\FloatTok{1.5}\NormalTok{, }\DataTypeTok{beta=}\OperatorTok{-}\FloatTok{0.06}\NormalTok{ ),}
 \KeywordTok{list}\NormalTok{( }\DataTypeTok{alpha=}\FloatTok{0.5}\NormalTok{, }\DataTypeTok{beta=}\OperatorTok{-}\FloatTok{0.04}\NormalTok{ ) )}
\NormalTok{ a2.dat <-}\StringTok{ }\KeywordTok{list}\NormalTok{( }\DataTypeTok{fatal=}\KeywordTok{c}\NormalTok{(airline}\OperatorTok{$}\NormalTok{fatal,}\OtherTok{NA}\NormalTok{),}
 \DataTypeTok{miles=}\KeywordTok{c}\NormalTok{(airline}\OperatorTok{$}\NormalTok{miles,}\DecValTok{20}\NormalTok{), }\DataTypeTok{I=}\DecValTok{27}\NormalTok{ )}
\NormalTok{ a2.par <-}\StringTok{ }\KeywordTok{c}\NormalTok{(}\StringTok{"alpha"}\NormalTok{,}\StringTok{"beta"}\NormalTok{,}\StringTok{"fatal[27]"}\NormalTok{)}
 \CommentTok{# Model compilation and burn-in}
\NormalTok{ a2.mod <-}\StringTok{ }\KeywordTok{jags.model}\NormalTok{( }\DataTypeTok{file =} \StringTok{"a2.jag"}\NormalTok{,}
 \DataTypeTok{data =}\NormalTok{ a2.dat,}
 \DataTypeTok{inits =}\NormalTok{ a2.ini,}
 \DataTypeTok{n.chains =} \DecValTok{3}\NormalTok{,}
 \DataTypeTok{n.adapt =} \DecValTok{1000}\NormalTok{ )}
\end{Highlighting}
\end{Shaded}

\begin{verbatim}
## Compiling model graph
##    Resolving undeclared variables
##    Allocating nodes
## Graph information:
##    Observed stochastic nodes: 26
##    Unobserved stochastic nodes: 3
##    Total graph size: 221
## 
## Initializing model
\end{verbatim}

\begin{Shaded}
\begin{Highlighting}[]
 \CommentTok{# Sampling from the posterior}
\NormalTok{ a2.res <-}\StringTok{ }\KeywordTok{coda.samples}\NormalTok{( a2.mod,}
 \DataTypeTok{var =}\NormalTok{ a2.par,}
 \DataTypeTok{n.iter =} \DecValTok{10000}\NormalTok{,}
 \DataTypeTok{thin =} \DecValTok{10}\NormalTok{ )}
 \KeywordTok{summary}\NormalTok{( a2.res )}
\end{Highlighting}
\end{Shaded}

\begin{verbatim}
## 
## Iterations = 1010:11000
## Thinning interval = 10 
## Number of chains = 3 
## Sample size per chain = 1000 
## 
## 1. Empirical mean and standard deviation for each variable,
##    plus standard error of the mean:
## 
##               Mean       SD  Naive SE Time-series SE
## alpha      1.17527 0.043791 7.995e-04      0.0007995
## beta      -0.06886 0.005291 9.659e-05      0.0001002
## fatal[27] 20.10200 4.719854 8.617e-02      0.0904746
## 
## 2. Quantiles for each variable:
## 
##               2.5%      25%      50%      75%    97.5%
## alpha      1.08903  1.14662  1.17493  1.20443  1.26011
## beta      -0.07891 -0.07247 -0.06885 -0.06536 -0.05841
## fatal[27] 12.00000 17.00000 20.00000 23.00000 30.00000
\end{verbatim}

\begin{Shaded}
\begin{Highlighting}[]
\KeywordTok{library}\NormalTok{( Epi )}
\end{Highlighting}
\end{Shaded}

\begin{verbatim}
## Warning: package 'Epi' was built under R version 3.5.2
\end{verbatim}

\begin{Shaded}
\begin{Highlighting}[]
\KeywordTok{ci.lin}\NormalTok{( glm2)}
\end{Highlighting}
\end{Shaded}

\begin{verbatim}
##                   Estimate      StdErr         z             P        2.5%
## (Intercept)     1.17611148 0.043199710  27.22499 3.287670e-163  1.09144161
## I(year - 1985) -0.06874189 0.005393721 -12.74480  3.332545e-37 -0.07931339
##                      97.5%
## (Intercept)     1.26078136
## I(year - 1985) -0.05817039
\end{verbatim}

we can see that beta and year are virtually identical according to the
CI seen here and the 95\% posterior interval from the bugs simulation

\begin{Shaded}
\begin{Highlighting}[]
\NormalTok{result2 =}\StringTok{ }\KeywordTok{ggs}\NormalTok{(a2.res)}
\KeywordTok{ggs_density}\NormalTok{(result2)}
\end{Highlighting}
\end{Shaded}

\includegraphics{reportPreview_files/figure-latex/unnamed-chunk-12-1.pdf}

\begin{Shaded}
\begin{Highlighting}[]
\KeywordTok{ggs_traceplot}\NormalTok{(result2)}
\end{Highlighting}
\end{Shaded}

\includegraphics{reportPreview_files/figure-latex/unnamed-chunk-12-2.pdf}

\begin{Shaded}
\begin{Highlighting}[]
\KeywordTok{ggs_running}\NormalTok{(result2)}
\end{Highlighting}
\end{Shaded}

\includegraphics{reportPreview_files/figure-latex/unnamed-chunk-12-3.pdf}

\begin{Shaded}
\begin{Highlighting}[]
\KeywordTok{ggs_autocorrelation}\NormalTok{(result2)}
\end{Highlighting}
\end{Shaded}

\includegraphics{reportPreview_files/figure-latex/unnamed-chunk-12-4.pdf}

\section{Model 3}\label{model-3}

\begin{Shaded}
\begin{Highlighting}[]
\KeywordTok{library}\NormalTok{(rjags)}
 \KeywordTok{cat}\NormalTok{( }\StringTok{"model}
\StringTok{ \{}
\StringTok{ for( i in 1:I )}
\StringTok{ \{}
\StringTok{ fatal[i] ~ dpois( mu[i] )}
\StringTok{ log(mu[i]) <- alpha + beta*miles[i]+ beta2 * miles[i]*rate[i]}
\StringTok{ \}}
\StringTok{ alpha ~ dnorm(0,0.0001)}
\StringTok{ beta ~ dnorm(0.0001,0.00001)}
\StringTok{ beta2 ~ dnorm(0.0001,0.00001)}
\StringTok{ \}"}\NormalTok{,}
 
 \DataTypeTok{file=}\StringTok{"a3.jag"}\NormalTok{ )}

\NormalTok{ a3.ini=}\StringTok{ }\KeywordTok{list}\NormalTok{(}
    \KeywordTok{list}\NormalTok{(}\DataTypeTok{alpha =} \FloatTok{0.01}\NormalTok{, }\DataTypeTok{beta =} \FloatTok{0.6}\NormalTok{, }\DataTypeTok{beta2=}\FloatTok{0.1}\NormalTok{),}
    \KeywordTok{list}\NormalTok{(}\DataTypeTok{alpha =} \FloatTok{0.2}\NormalTok{, }\DataTypeTok{beta =} \FloatTok{0.2}\NormalTok{, }\DataTypeTok{beta2=}\FloatTok{0.2}\NormalTok{),}
    \KeywordTok{list}\NormalTok{(}\DataTypeTok{alpha =} \FloatTok{0.7}\NormalTok{, }\DataTypeTok{beta =} \FloatTok{0.7}\NormalTok{, }\DataTypeTok{beta2=}\FloatTok{0.3}\NormalTok{))}
\NormalTok{ a3.dat <-}\StringTok{ }\KeywordTok{list}\NormalTok{( }\DataTypeTok{rate=} \KeywordTok{c}\NormalTok{(airline}\OperatorTok{$}\NormalTok{rate,}\FloatTok{0.7080}\NormalTok{,}\FloatTok{0.3}\NormalTok{,}\FloatTok{0.4}\NormalTok{),}\DataTypeTok{fatal=}\KeywordTok{c}\NormalTok{(airline}\OperatorTok{$}\NormalTok{fatal,}\OtherTok{NA}\NormalTok{,}\OtherTok{NA}\NormalTok{,}\OtherTok{NA}\NormalTok{), }\DataTypeTok{miles=}\KeywordTok{c}\NormalTok{(airline}\OperatorTok{$}\NormalTok{miles,}\FloatTok{19.775}\NormalTok{,}\FloatTok{23.300}\NormalTok{,}\FloatTok{20.3}\NormalTok{), }\DataTypeTok{I=}\DecValTok{29}\NormalTok{ )}
\NormalTok{ a3.par <-}\StringTok{ }\KeywordTok{c}\NormalTok{(}\StringTok{"alpha"}\NormalTok{,}\StringTok{"beta"}\NormalTok{,}\StringTok{"beta2"}\NormalTok{,}\StringTok{"fatal[27]"}\NormalTok{,}\StringTok{"fatal[28]"}\NormalTok{,}\StringTok{"fatal[29]"}\NormalTok{)}

 \CommentTok{# Model compilation and burn-in}
\NormalTok{ a3.mod <-}\StringTok{ }\KeywordTok{jags.model}\NormalTok{(}\DataTypeTok{file =} \StringTok{"a3.jag"}\NormalTok{,}\DataTypeTok{data =}\NormalTok{ a3.dat,}\DataTypeTok{inits =}\NormalTok{ a3.ini,}\DataTypeTok{n.chains =} \DecValTok{3}\NormalTok{,}\DataTypeTok{n.adapt =} \DecValTok{1000}\NormalTok{)}
\end{Highlighting}
\end{Shaded}

\begin{verbatim}
## Compiling model graph
##    Resolving undeclared variables
##    Allocating nodes
## Graph information:
##    Observed stochastic nodes: 26
##    Unobserved stochastic nodes: 6
##    Total graph size: 210
## 
## Initializing model
\end{verbatim}

\begin{Shaded}
\begin{Highlighting}[]
 \KeywordTok{update}\NormalTok{(a3.mod,}\DecValTok{1000}\NormalTok{)}
 \CommentTok{# Sampling from the posterior}
\NormalTok{ a33.res <-}\StringTok{ }\KeywordTok{coda.samples}\NormalTok{( a3.mod,}\DataTypeTok{var =}\NormalTok{ a3.par,}\DataTypeTok{n.iter =} \DecValTok{10000}\NormalTok{,}\DataTypeTok{thin =} \DecValTok{50}\NormalTok{ )}
 \KeywordTok{summary}\NormalTok{( a33.res )}
\end{Highlighting}
\end{Shaded}

\begin{verbatim}
## 
## Iterations = 2050:12000
## Thinning interval = 50 
## Number of chains = 3 
## Sample size per chain = 200 
## 
## 1. Empirical mean and standard deviation for each variable,
##    plus standard error of the mean:
## 
##                Mean       SD  Naive SE Time-series SE
## alpha      2.182434 0.261671 0.0106827      0.0166427
## beta      -0.001854 0.008665 0.0003538      0.0003538
## beta2      0.041448 0.008916 0.0003640      0.0005320
## fatal[27] 15.635000 4.516167 0.1843718      0.1842844
## fatal[28] 11.686667 3.980984 0.1625230      0.1824109
## fatal[29] 12.178333 3.952962 0.1613790      0.1611767
## 
## 2. Quantiles for each variable:
## 
##               2.5%       25%       50%       75%    97.5%
## alpha      1.67354  2.017516  2.177347  2.356533  2.68220
## beta      -0.01853 -0.007751 -0.001673  0.004757  0.01386
## beta2      0.02365  0.035342  0.041685  0.047202  0.05813
## fatal[27]  7.00000 13.000000 15.000000 19.000000 25.00000
## fatal[28]  5.00000  9.000000 11.000000 14.000000 20.00000
## fatal[29]  6.00000  9.000000 12.000000 15.000000 21.00000
\end{verbatim}

\begin{Shaded}
\begin{Highlighting}[]
 \CommentTok{#predictions}
\end{Highlighting}
\end{Shaded}

\subsection{fernando}\label{fernando}

\begin{Shaded}
\begin{Highlighting}[]
\KeywordTok{library}\NormalTok{(rjags)}
 \KeywordTok{cat}\NormalTok{( }\StringTok{"model}
\StringTok{ \{}
\StringTok{ for( i in 1:I )}
\StringTok{ \{}
\StringTok{ fatal[i] ~ dpois( mu[i] )}
\StringTok{ log(mu[i]) <- alpha + beta2 * miles[i]*rate[i]}
\StringTok{ \}}
\StringTok{ alpha ~ dnorm(0,0.0001)}
\StringTok{ }
\StringTok{ beta2 ~ dnorm(0.0001,0.00001)}
\StringTok{ \}"}\NormalTok{,}
 
 \DataTypeTok{file=}\StringTok{"a3.jag"}\NormalTok{ )}

\NormalTok{ a3.ini=}\StringTok{ }\KeywordTok{list}\NormalTok{(}
    \KeywordTok{list}\NormalTok{(}\DataTypeTok{alpha =} \FloatTok{0.01}\NormalTok{, }\DataTypeTok{beta2=}\FloatTok{0.1}\NormalTok{),}
    \KeywordTok{list}\NormalTok{(}\DataTypeTok{alpha =} \FloatTok{0.2}\NormalTok{, }\DataTypeTok{beta2=}\FloatTok{0.2}\NormalTok{),}
    \KeywordTok{list}\NormalTok{(}\DataTypeTok{alpha =} \FloatTok{0.7}\NormalTok{,  }\DataTypeTok{beta2=}\FloatTok{0.3}\NormalTok{))}
\NormalTok{ a3.dat <-}\StringTok{ }\KeywordTok{list}\NormalTok{( }\DataTypeTok{rate=} \KeywordTok{c}\NormalTok{(airline}\OperatorTok{$}\NormalTok{rate,}\FloatTok{0.7080}\NormalTok{,}\FloatTok{0.3}\NormalTok{,}\FloatTok{0.4}\NormalTok{),}\DataTypeTok{fatal=}\KeywordTok{c}\NormalTok{(airline}\OperatorTok{$}\NormalTok{fatal,}\OtherTok{NA}\NormalTok{,}\OtherTok{NA}\NormalTok{,}\OtherTok{NA}\NormalTok{), }\DataTypeTok{miles=}\KeywordTok{c}\NormalTok{(airline}\OperatorTok{$}\NormalTok{miles,}\FloatTok{19.775}\NormalTok{,}\FloatTok{23.300}\NormalTok{,}\FloatTok{20.3}\NormalTok{), }\DataTypeTok{I=}\DecValTok{29}\NormalTok{ )}
\NormalTok{ a3.par <-}\StringTok{ }\KeywordTok{c}\NormalTok{(}\StringTok{"alpha"}\NormalTok{,}\StringTok{"beta2"}\NormalTok{,}\StringTok{"fatal[27]"}\NormalTok{,}\StringTok{"fatal[28]"}\NormalTok{,}\StringTok{"fatal[29]"}\NormalTok{)}

 \CommentTok{# Model compilation and burn-in}
\NormalTok{ a3.mod <-}\StringTok{ }\KeywordTok{jags.model}\NormalTok{(}\DataTypeTok{file =} \StringTok{"a3.jag"}\NormalTok{,}\DataTypeTok{data =}\NormalTok{ a3.dat,}\DataTypeTok{inits =}\NormalTok{ a3.ini,}\DataTypeTok{n.chains =} \DecValTok{3}\NormalTok{,}\DataTypeTok{n.adapt =} \DecValTok{1000}\NormalTok{)}
\end{Highlighting}
\end{Shaded}

\begin{verbatim}
## Compiling model graph
##    Resolving undeclared variables
##    Allocating nodes
## Graph information:
##    Observed stochastic nodes: 26
##    Unobserved stochastic nodes: 5
##    Total graph size: 180
## 
## Initializing model
\end{verbatim}

\begin{Shaded}
\begin{Highlighting}[]
 \KeywordTok{update}\NormalTok{(a3.mod,}\DecValTok{1000}\NormalTok{)}
 \CommentTok{# Sampling from the posterior}
\NormalTok{ a3.res <-}\StringTok{ }\KeywordTok{coda.samples}\NormalTok{( a3.mod,}\DataTypeTok{var =}\NormalTok{ a3.par,}\DataTypeTok{n.iter =} \DecValTok{10000}\NormalTok{,}\DataTypeTok{thin =} \DecValTok{50}\NormalTok{ )}
 \KeywordTok{summary}\NormalTok{( a3.res )}
\end{Highlighting}
\end{Shaded}

\begin{verbatim}
## 
## Iterations = 2050:12000
## Thinning interval = 50 
## Number of chains = 3 
## Sample size per chain = 200 
## 
## 1. Empirical mean and standard deviation for each variable,
##    plus standard error of the mean:
## 
##               Mean       SD  Naive SE Time-series SE
## alpha      2.11421 0.226334 0.0092400      0.0135316
## beta2      0.04341 0.008728 0.0003563      0.0004975
## fatal[27] 15.29333 4.311412 0.1760126      0.1759037
## fatal[28] 11.39500 3.689398 0.1506191      0.1815618
## fatal[29] 11.98667 4.071132 0.1662033      0.1727486
## 
## 2. Quantiles for each variable:
## 
##             2.5%      25%     50%     75%    97.5%
## alpha     1.6881  1.95110  2.1213  2.2657  2.54912
## beta2     0.0263  0.03725  0.0432  0.0494  0.05936
## fatal[27] 7.0000 12.00000 15.0000 18.0000 24.00000
## fatal[28] 4.9750  9.00000 11.0000 14.0000 18.02500
## fatal[29] 5.0000  9.00000 12.0000 14.2500 21.00000
\end{verbatim}

\begin{Shaded}
\begin{Highlighting}[]
 \CommentTok{#predictions}
\end{Highlighting}
\end{Shaded}

\begin{Shaded}
\begin{Highlighting}[]
\NormalTok{result3 =}\StringTok{ }\KeywordTok{ggs}\NormalTok{(a3.res)}
\KeywordTok{ggs_density}\NormalTok{(result3)}
\end{Highlighting}
\end{Shaded}

\includegraphics{reportPreview_files/figure-latex/unnamed-chunk-13-1.pdf}

\begin{Shaded}
\begin{Highlighting}[]
\KeywordTok{ggs_traceplot}\NormalTok{(result3)}
\end{Highlighting}
\end{Shaded}

\includegraphics{reportPreview_files/figure-latex/unnamed-chunk-13-2.pdf}

\begin{Shaded}
\begin{Highlighting}[]
\KeywordTok{ggs_running}\NormalTok{(result3)}
\end{Highlighting}
\end{Shaded}

\includegraphics{reportPreview_files/figure-latex/unnamed-chunk-13-3.pdf}

\begin{Shaded}
\begin{Highlighting}[]
\KeywordTok{ggs_autocorrelation}\NormalTok{(result3)}
\end{Highlighting}
\end{Shaded}

\includegraphics{reportPreview_files/figure-latex/unnamed-chunk-13-4.pdf}

\section{Model 4 test}\label{model-4-test}

\begin{Shaded}
\begin{Highlighting}[]
\KeywordTok{library}\NormalTok{(rjags)}
 \KeywordTok{cat}\NormalTok{( }\StringTok{"model}
\StringTok{ \{}
\StringTok{ for( i in 1:I )}
\StringTok{ \{}
\StringTok{ fatal[i] ~ dnorm( mu[i],1/mu[i] )}
\StringTok{ mu[i] <- alpha + beta2 * miles[i]}
\StringTok{ \}}
\StringTok{ alpha ~ dnorm(0,0.0001)}
\StringTok{ beta2 ~ dnorm(0,0.0001)}
\StringTok{ \}"}\NormalTok{,}
 \DataTypeTok{file=}\StringTok{"a4.jag"}\NormalTok{ )}

\NormalTok{ a4.ini=}\StringTok{ }\KeywordTok{list}\NormalTok{(}
    \KeywordTok{list}\NormalTok{(}\DataTypeTok{alpha =} \FloatTok{0.01}\NormalTok{, }\DataTypeTok{beta2=}\DecValTok{3}\NormalTok{),}
    \KeywordTok{list}\NormalTok{(}\DataTypeTok{alpha =} \FloatTok{0.2}\NormalTok{, }\DataTypeTok{beta2=}\DecValTok{1}\NormalTok{),}
    \KeywordTok{list}\NormalTok{(}\DataTypeTok{alpha =} \FloatTok{0.7}\NormalTok{, }\DataTypeTok{beta2=}\DecValTok{2}\NormalTok{))}
\NormalTok{ a4.dat <-}\StringTok{ }\KeywordTok{list}\NormalTok{( }\DataTypeTok{rate =} \KeywordTok{c}\NormalTok{(airline}\OperatorTok{$}\NormalTok{rate,}\FloatTok{0.7080}\NormalTok{,}\FloatTok{0.3}\NormalTok{),}\DataTypeTok{fatal=}\KeywordTok{c}\NormalTok{(airline}\OperatorTok{$}\NormalTok{fatal,}\OtherTok{NA}\NormalTok{,}\OtherTok{NA}\NormalTok{), }\DataTypeTok{miles=}\KeywordTok{c}\NormalTok{(airline}\OperatorTok{$}\NormalTok{miles,}\FloatTok{19.775}\NormalTok{,}\FloatTok{23.300}\NormalTok{), }\DataTypeTok{I=}\DecValTok{28}\NormalTok{ )}
\NormalTok{ a4.par <-}\StringTok{ }\KeywordTok{c}\NormalTok{(}\StringTok{"alpha"}\NormalTok{,}\StringTok{"beta2"}\NormalTok{,}\StringTok{"fatal[27]"}\NormalTok{,}\StringTok{"fatal[28]"}\NormalTok{)}

 \CommentTok{# Model compilation and burn-in}
\NormalTok{ a4.mod <-}\StringTok{ }\KeywordTok{jags.model}\NormalTok{(}\DataTypeTok{file =} \StringTok{"a4.jag"}\NormalTok{,}\DataTypeTok{data =}\NormalTok{ a4.dat,}\DataTypeTok{inits =}\NormalTok{ a4.ini,}\DataTypeTok{n.chains =} \DecValTok{3}\NormalTok{,}\DataTypeTok{n.adapt =} \DecValTok{1000}\NormalTok{ )}
\end{Highlighting}
\end{Shaded}

\begin{verbatim}
## Warning in jags.model(file = "a4.jag", data = a4.dat, inits = a4.ini,
## n.chains = 3, : Unused variable "rate" in data
\end{verbatim}

\begin{verbatim}
## Compiling model graph
##    Resolving undeclared variables
##    Allocating nodes
## Graph information:
##    Observed stochastic nodes: 26
##    Unobserved stochastic nodes: 4
##    Total graph size: 146
## 
## Initializing model
\end{verbatim}

\begin{Shaded}
\begin{Highlighting}[]
 \CommentTok{# Sampling from the posterior}
 \KeywordTok{update}\NormalTok{(a4.mod,}\DecValTok{1000}\NormalTok{)}
\NormalTok{ a4.res <-}\StringTok{ }\KeywordTok{coda.samples}\NormalTok{( a4.mod,}\DataTypeTok{var =}\NormalTok{ a4.par,}\DataTypeTok{n.iter =} \DecValTok{3000}\NormalTok{,}\DataTypeTok{thin =} \DecValTok{10}\NormalTok{ )}
 \KeywordTok{summary}\NormalTok{( a4.res )}
\end{Highlighting}
\end{Shaded}

\begin{verbatim}
## 
## Iterations = 2010:5000
## Thinning interval = 10 
## Number of chains = 3 
## Sample size per chain = 300 
## 
## 1. Empirical mean and standard deviation for each variable,
##    plus standard error of the mean:
## 
##              Mean     SD Naive SE Time-series SE
## alpha     27.6432 2.5634 0.085446        0.12360
## beta2     -0.3056 0.2248 0.007493        0.01049
## fatal[27] 21.7669 5.0210 0.167366        0.16753
## fatal[28] 20.5715 5.5941 0.186470        0.19410
## 
## 2. Quantiles for each variable:
## 
##             2.5%     25%     50%     75%   97.5%
## alpha     22.701 25.9546 27.6330 29.3264 32.7572
## beta2     -0.716 -0.4613 -0.3126 -0.1615  0.1443
## fatal[27] 12.204 18.2801 21.7156 25.1187 31.6575
## fatal[28] 10.858 16.4191 20.1038 24.3395 31.9747
\end{verbatim}

\begin{Shaded}
\begin{Highlighting}[]
 \CommentTok{#predictions}
\end{Highlighting}
\end{Shaded}

\begin{Shaded}
\begin{Highlighting}[]
\NormalTok{result4 =}\StringTok{ }\KeywordTok{ggs}\NormalTok{(a4.res)}
\KeywordTok{ggs_density}\NormalTok{(result4)}
\end{Highlighting}
\end{Shaded}

\includegraphics{reportPreview_files/figure-latex/unnamed-chunk-14-1.pdf}

\begin{Shaded}
\begin{Highlighting}[]
\KeywordTok{ggs_traceplot}\NormalTok{(result4)}
\end{Highlighting}
\end{Shaded}

\includegraphics{reportPreview_files/figure-latex/unnamed-chunk-14-2.pdf}

\begin{Shaded}
\begin{Highlighting}[]
\KeywordTok{ggs_running}\NormalTok{(result4)}
\end{Highlighting}
\end{Shaded}

\includegraphics{reportPreview_files/figure-latex/unnamed-chunk-14-3.pdf}

\begin{Shaded}
\begin{Highlighting}[]
\KeywordTok{ggs_autocorrelation}\NormalTok{(result4)}
\end{Highlighting}
\end{Shaded}

\includegraphics{reportPreview_files/figure-latex/unnamed-chunk-14-4.pdf}

\begin{Shaded}
\begin{Highlighting}[]
\NormalTok{ a3.m <-}\StringTok{ }\KeywordTok{as.matrix}\NormalTok{(a33.res)}
\NormalTok{log.enum.}\DecValTok{2002}\NormalTok{ <-}\StringTok{ }\NormalTok{a3.m[,}\StringTok{"alpha"}\NormalTok{] }\OperatorTok{+}\StringTok{ }\NormalTok{a3.m[,}\StringTok{"beta"}\NormalTok{]}\OperatorTok{*}\FloatTok{19.775}
\CommentTok{#print(exp(log.enum.2002))}
\KeywordTok{summary}\NormalTok{( }\KeywordTok{exp}\NormalTok{(log.enum.}\DecValTok{2002}\NormalTok{ ))}
\end{Highlighting}
\end{Shaded}

\begin{verbatim}
##    Min. 1st Qu.  Median    Mean 3rd Qu.    Max. 
##   3.919   7.332   8.630   8.782   9.900  16.905
\end{verbatim}

\begin{Shaded}
\begin{Highlighting}[]
\NormalTok{v =}\KeywordTok{exp}\NormalTok{(log.enum.}\DecValTok{2002}\NormalTok{ )}
\CommentTok{#plot(density(v))}
\NormalTok{(e2002.qnt <-}\StringTok{ }\KeywordTok{quantile}\NormalTok{( v, }\DataTypeTok{probs=}\KeywordTok{c}\NormalTok{(}\DecValTok{50}\NormalTok{,}\FloatTok{2.5}\NormalTok{,}\FloatTok{97.5}\NormalTok{)}\OperatorTok{/}\DecValTok{100}\NormalTok{))}
\end{Highlighting}
\end{Shaded}

\begin{verbatim}
##       50%      2.5%     97.5% 
##  8.629934  5.542588 13.671873
\end{verbatim}

\begin{Shaded}
\begin{Highlighting}[]
\KeywordTok{plot}\NormalTok{( }\KeywordTok{density}\NormalTok{(v), }\DataTypeTok{type=}\StringTok{"l"}\NormalTok{, }\DataTypeTok{lwd=}\DecValTok{2}\NormalTok{, }\DataTypeTok{col=} \StringTok{"red"}\NormalTok{ )}
\KeywordTok{abline}\NormalTok{( }\DataTypeTok{c=}\NormalTok{e2002.qnt)}
\end{Highlighting}
\end{Shaded}

\includegraphics{reportPreview_files/figure-latex/unnamed-chunk-15-1.pdf}

\subsection{Prediction}\label{prediction}

We want to predict the expected number of airlines fatalities in 2002
(assuming that the amount of miles flown is \$ 20\times 10\^{}\{12\}\$
), so we need the posterior of
\(exp( \alpha + \beta \times (2002-1985))\times 20\)

\begin{Shaded}
\begin{Highlighting}[]
\NormalTok{a2.m <-}\StringTok{ }\KeywordTok{as.matrix}\NormalTok{(a2.res)}
\NormalTok{enum.}\DecValTok{2002}\NormalTok{ <-}\StringTok{ }\KeywordTok{exp}\NormalTok{(a2.m[,}\StringTok{"alpha"}\NormalTok{] }\OperatorTok{+}\StringTok{ }\NormalTok{a2.m[,}\StringTok{"beta"}\NormalTok{]}\OperatorTok{*}\DecValTok{17}\NormalTok{)}\OperatorTok{*}\DecValTok{20}
\KeywordTok{summary}\NormalTok{( enum.}\DecValTok{2002}\NormalTok{ )}
\end{Highlighting}
\end{Shaded}

\begin{verbatim}
##    Min. 1st Qu.  Median    Mean 3rd Qu.    Max. 
##   14.44   19.00   20.07   20.16   21.25   26.45
\end{verbatim}

\begin{Shaded}
\begin{Highlighting}[]
\NormalTok{(e2002.qnt <-}\StringTok{ }\KeywordTok{quantile}\NormalTok{( enum.}\DecValTok{2002}\NormalTok{, }\DataTypeTok{probs=}\KeywordTok{c}\NormalTok{(}\DecValTok{50}\NormalTok{,}\FloatTok{2.5}\NormalTok{,}\FloatTok{97.5}\NormalTok{)}\OperatorTok{/}\DecValTok{100}\NormalTok{))}
\end{Highlighting}
\end{Shaded}

\begin{verbatim}
##      50%     2.5%    97.5% 
## 20.07047 17.09753 23.64528
\end{verbatim}

\begin{Shaded}
\begin{Highlighting}[]
\KeywordTok{plot}\NormalTok{( }\KeywordTok{density}\NormalTok{(enum.}\DecValTok{2002}\NormalTok{), }\DataTypeTok{type=}\StringTok{"l"}\NormalTok{, }\DataTypeTok{lwd=}\DecValTok{2}\NormalTok{, }\DataTypeTok{col=} \StringTok{"red"}\NormalTok{ )}
\KeywordTok{abline}\NormalTok{( }\DataTypeTok{v=}\NormalTok{e2002.qnt)}
\end{Highlighting}
\end{Shaded}

\includegraphics{reportPreview_files/figure-latex/unnamed-chunk-16-1.pdf}

The node fatal{[}27{]} contains the predictive distribution for the
number of fatal accidents in 2002. Its posterior mean is 20.04 (similar
to that for the expected number of fatal accidents in 2002) with a
standard deviation of 4.864 and 95\% interval {[}11,30{]} We can plot
its distribution.

\begin{Shaded}
\begin{Highlighting}[]
\KeywordTok{plot}\NormalTok{( }\KeywordTok{table}\NormalTok{(a2.m[,}\StringTok{"fatal[27]"}\NormalTok{]), }\DataTypeTok{type=}\StringTok{"h"}\NormalTok{, }\DataTypeTok{lwd=}\DecValTok{5}\NormalTok{, }\DataTypeTok{lend=}\DecValTok{2}\NormalTok{, }\DataTypeTok{col=}\KeywordTok{gray}\NormalTok{(}\FloatTok{0.5}\NormalTok{), }\DataTypeTok{bty=}\StringTok{"n"}\NormalTok{, }\DataTypeTok{ylab=}\StringTok{""}\NormalTok{ )}
\end{Highlighting}
\end{Shaded}

\includegraphics{reportPreview_files/figure-latex/unnamed-chunk-17-1.pdf}

We can notice from the table of true values bolow that our prediction of
20 for the number of fatal accidents in 2002 was wrong as the right one
was 14 but still fits well within the prediction interval of (11,30).
Finally we notice also that, since 1976, the rate of fatal accident per
air mile flown has decreased becoming ten times lower.

\begin{verbatim}
  Year      Fatal accidents   Passenger miles    Accident rate
               

  2002      14                19.775             0.7080
  2003      7                 23.300             0.3004
  2004      9                 20.300             0.4433
\end{verbatim}

\subsection{Model comparison throughout DIC
index}\label{model-comparison-throughout-dic-index}

A good way to select the model that fits best our data, is using the DIC
measure. The DIC (deviance information criterion) is a Bayesian
criterion for model comparison, defined as:
\[ DIC = D(\bar{\theta}) + 2_{p_D}\] Where:
\(D(\theta) = -2logL(\theta)\), \(p_D = \bar D - D(\bar \theta)\), is
the effective number of paramenters of the model, the larger \(p_D\) the
easier the model fits on the data. \(\bar D = E[D(\theta)]\), its the
mean deviance. is a measure of how well the model fits the data, the
larger the worse the fit. \(D(\bar \theta) = D[E(\theta)]\)

the basic idea is that the models with smaller DIC should be preferred.
Models are penalized both by the value of \(\bar D\), which favors a
good fit, but also by the effective number of parameters \(p_D\). Since
\(\bar D\) will decrease as the number of parameters in a model
increases, the \(p_{D}\) term compensates for this effect by favoring
models with a smaller number of parameters.

\begin{Shaded}
\begin{Highlighting}[]
\CommentTok{#Model Comparison}
\NormalTok{DICmod1 =}\StringTok{ }\KeywordTok{dic.samples}\NormalTok{(a1.mod, }\DataTypeTok{n.iter=}\DecValTok{10000}\NormalTok{, }\DataTypeTok{thin=}\DecValTok{10}\NormalTok{, }\DataTypeTok{type=}\StringTok{"pD"}\NormalTok{)}
\NormalTok{DICmod2 =}\StringTok{ }\KeywordTok{dic.samples}\NormalTok{(a2.mod, }\DataTypeTok{n.iter=}\DecValTok{10000}\NormalTok{, }\DataTypeTok{thin =}\DecValTok{10}\NormalTok{, }\DataTypeTok{type=}\StringTok{"pD"}\NormalTok{)}
\KeywordTok{print}\NormalTok{(}\StringTok{'model 1 DIC'}\NormalTok{)}
\end{Highlighting}
\end{Shaded}

\begin{verbatim}
## [1] "model 1 DIC"
\end{verbatim}

\begin{Shaded}
\begin{Highlighting}[]
\NormalTok{DICmod1}
\end{Highlighting}
\end{Shaded}

\begin{verbatim}
## Mean deviance:  156.2 
## penalty 1.018 
## Penalized deviance: 157.3
\end{verbatim}

\begin{Shaded}
\begin{Highlighting}[]
\KeywordTok{print}\NormalTok{(}\StringTok{'model 2 DIC'}\NormalTok{)}
\end{Highlighting}
\end{Shaded}

\begin{verbatim}
## [1] "model 2 DIC"
\end{verbatim}

\begin{Shaded}
\begin{Highlighting}[]
\NormalTok{DICmod2}
\end{Highlighting}
\end{Shaded}

\begin{verbatim}
## Mean deviance:  155 
## penalty 1.977 
## Penalized deviance: 157
\end{verbatim}


\end{document}
